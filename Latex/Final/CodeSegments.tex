\chapter{Code Segments}\label{app:CodeSegments}

\section{Fibonacci in C}\label{sec:FibC}

This code segment shows an example for constructing the fibonacci sequence in C in 2 ways: without recursion, and with recursion.

\begin{verbatim}
    //Given an integer n, calculate the first n numbers
    //of the fibonacci sequence without recursion
    
    void sequentialFibonacci (int n) {
        if (n < 1) {
            printf("Input must be an integer greater than 0");
                return;
        }
        int i = 1;
        int sub1 = 0;
        int sub2 = 1;
        for (i = 1; i <= n; i+=1) {
            if (i > 2) {
                int curr = sub1 + sub2;
                sub1 = sub2;
                sub2 = curr;
                printf("%d ", curr);
            }
            else if (i == 1) {
                printf("%d ", sub1);
            }
            else if (i == 2) {
                printf("%d ", sub2);
            }
        }
    }

    /***************************************************************/

    //Recursive case

    //keep track of current Index, given amount of fibonacci numbers
    //to print, and propogate the two subnumbers to the next step
    void recursiveFibonacci (int currIndex, int n, int sub1, int sub2) {
        if (currIndex < n) {
            printf("%d ", sub1 + sub2);
            recursiveFibonacci(currIndex + 1, n, sub2, sub1 + sub2);
        }
        return;
    }

    //handle base cases (exit conditions)
    //otherwise start the recursive process
    void startRecursiveFibonacci (int n) {
        if (n < 1) {
            printf("Input must be an integer greater than 0");
        } else if (n == 1) {
            printf("%d ", 0);
        } else if (n == 2) {
            printf("%d %d ", 0, 1);
        } else {
            printf("%d %d ", 0, 1);
            recursiveFibonacci(0, n - 2, 0, 1);
        }
        return;
    }
\end{verbatim}

\section{Java School Example}\label{sec:JavaSchool}

Java code to demonstrate OOP using the school example mentioned in \ref{subsubsec:OOPL}


\begin{verbatim}
    class School {
        private int numEnrolledStudents;
        
        public School () {
            this.numEnrolledStudents = 0;
        }
        
        public School (int numAlreadyEnrolled) {
            this.numEnrolledStudents = numAlreadyEnrolled;
        }
        
        public int getNumEnrolled() {
            return this.numEnrolledStudents;
        }
        
        public void setNumEnrolled(int numStudents) {
            this.numEnrolledStudents = numStudents;
            return;
        }
    }
    
    class RunCode {
        public static void main(String[] args) {
            School NewSchool = new School();
            
            System.out.println("The number of students enrolled 
                in the new school is: " + NewSchool.getNumEnrolled()); // 0
            
            NewSchool.setNumEnrolled(100);
            
            System.out.println("The number of students enrolled
                in the new school is: " + NewSchool.getNumEnrolled()); // 100
            
            School CSUN = new School(32172);
            
            System.out.println("The number of students enrolled
                at CSUN is: " + CSUN.getNumEnrolled());                // 32172
        }
    }
\end{verbatim}