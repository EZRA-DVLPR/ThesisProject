%\chapter{Different Proof Techniques for demonstrating Turing Completeness}
\chapter{Different Approaches for Proofs to Demonstrate Turing Completeness}

\section{Overview}

We will be exploring the different approaches to demonstrate TC for different systems.
I have oultined the approaches based on their respective discipline.
I hope that this allows for the logic to be legible and digestible.
My intention is to add clarity on where the logic behind these proofs/techniques exist.
For example, in the Computer Engineering perspective, they analyze the TM based on its mechanical properties such as Addition, Logic Gates and so forth.
This is vastly different compared to how Mathematicians show TM and TC, which is through the use of Lambda calculus which is a model for representing functions as a basic unit.
However, all proofs are equivalent in goal.
These are not the only perspectives and types of proofs for showing TC, as this is simply a survey into what TC looks like across the disciplines.

\section{Computer Engineering}

In this section, we will analyze what a TM looks like from a physical perspective.
This may seem contradictory because the TM is described as a theoeretical machine.
But in fact, the very computers that we use today are are capable of processing TC systems through the usage of programming languages.
This means that they are limited TMs, because they are bounded only in memory.
In this approach, we will look at the core components of Computer Engineering such as the wiring components, logic gates, adders, and so forth.

\subsection{Logical Design of a TM}

To define what a TM does, we must explore what it is capable of.
Recall the Church-Turing Thesis in \ref{subsec:Church-Turing Thesis}, "Every effectively calculable function can be computed by a Turing Machine."
Every effictively calculable function, as Turing and Church understood, was any mathematical calculation.
This means that a TM must have some ability to perform any operation on numbers, such as the basic operations of addition, subtraction, multiplication, and division.
Furthermore, they must be capable of combining these together to form more complex operations such as exponential arithmetic.
Beyond the mathematical aspect, they must allow for some sort of logic handling.
This can be achieved with OR, NOR, AND, etc. gates.

[BIOLOGY PAPER ON TM]

I will now give a succinct description of each of the necessary parts.

Adder
OR
NOR
AND
XOR
...
Multiplexer
Demultiplexer

\subsection{Constructing the TM}

Combining these smaller components to form a circuit.

\section{Computer Science}

\subsection{State Machines}

\subsubsection{Formal Technical Writing}

$\lambda \gamma \tau $ etc. being used to describe TC

\subsubsection{Gamified Writing}

$\lambda \gamma \tau $ etc. being used to describe TC, but based on a very verbose and interactive style

\subsection{Software Implementation}

Demonstrate an example of a problem/scenario which shows TC.

\subsubsection{Conway's Game of Life}

Classic example to implement

\subsubsection{Rule 110}

Simpler example to implement

\subsubsection{Calculator with Store Value}

More challenging to implement than the previous two.

\subsection{Interpreter for a known Turing Complete language}

Implement an interpreter for a known TC language in a given language.
Must faithfully and accurately simulate the behavior of the given language to be interpreted.

typically, we will implement a simple TC language such as brainfuck

\section{Mathematics}

\subsection{Lambda Calculus}

Basically a simplistic view of math functions.
can be shown as TC through construction of certain concepts.