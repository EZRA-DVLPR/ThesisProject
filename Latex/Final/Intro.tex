\chapter{Introduction}
%Background information.
%catch the audience up to speed with this part of the document

\section{Turing Machines}

Alan Turing is generally considered the father of computer science for his numerous contributions including: formalization of computation theory, algorithm design, complexity theory, as well as creating the idea of the Turing Machine.
A Turing machine can be described as a machine/automata that is capable of performing operations towards some desired goal given an input.
In a sense, it was designed to be capable of performing any single computable task, such as addition, division, concatenating strings, rendering graphics, etc.
TMs are at the highest level of computational power, i.e. capable of handling any computation.

INSERT DIAGRAM OF A SIMPLE TURING MACHINE THAT PERFORMS BINARY ADDITION

\subsection{Oracles}

Of course there also exists the Oracle, sometimes called Turing Machines with an oracle, which is capable of solving problems that TMs cannot.
It does so by having the ability to respond to any given problem from the TM it is connected to.
For example, the oracle would be able to solve the Halting Problem for the associated Turing Machine, but not the Halting Problem in general for all Turing Machines.

INSERT DIAGRAM OF A TM W/ AN ORACLE SOLVING THE HALTING PROBLEM

The reason the oracle is not considered more powerful is because in practice (i.e. reality), is because there is no such all-knowing source to retrieve information from.
As a result, I will look only at TMs for the rest of the thesis.

\subsection{Universal Turing Machines}

A simple abstraction of the standard TM is a Universal Turing Machine. A UTM is capable of solving any computable problem, given the exact process/rules of the TM that will solve it.
In essence it is a machine that is not hard-coded with what to perform when given input.
The UTM will read the input, and respond based on the rules given.
As a result, the UTM is equivalently as powerful as a TM.
The only functional difference is the usability of the UTM towards a larger number of problems as opposed to the TM being created for a singular problem.

INSERT DIAGRAM OF A SIMPLE UNIVERSAL TURING MACHINE THAT PERFORMS BINARY ADDITION
basically it just takes another input which are the instructions

\subsection{The Church-Turing Thesis}

According to the Church-Turing Thesis, every effectively calculable function can be computed by a Turing Machine.
As explained by Robin Gandy, this idea was to show that the limits of computation of reality are bounded by TMs.
Thus any automata that violates it, such as an Oracle Machine, would be capable of calculating a non-computable function.
In his words, displaying "Free Will".

[INSERT REFERENCE TO GANDY 1980 PAPER PG 123-148].

Generally speaking however, the consensus is that the Church-Turing Thesis is widely accepted as fact, and no such oracle machines exist as previously stated.

\section{Turing Complete}

Turing Complete is a closely related term when discussing Turing Machines.
For a system to be Turing Complete, it must be capable of performing any computation that a standard TM can perform.
An equivalent description would be that for a system to be TC, it must simulate a UTM.
By transitivity, if any system is proven to be TC, then it must be equivalent in power to all other systems that are TC.
Thus, all are TMs, which in turn means they are all equivalently the most powerful computation machine.

Some classical/practical examples of TC

Link some funny/interesting examples of TC that have been proven


In this thesis, we will look closely at the several methods to showing that a given system is TC.