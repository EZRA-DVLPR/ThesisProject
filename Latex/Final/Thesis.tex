\documentclass[12pt]{report} % use report for thesis formatting and setup with 12 pt font
\usepackage[american]{babel} % use standard american english for hyphenation rules, date format, etc.
\usepackage{mathptmx} % Times New Roman
\usepackage{graphicx} % for importing images
\usepackage{cite} % referencing citations in bib
\usepackage[doublespacing]{setspace} %\doublespacing % set double spaced for entire doc
\usepackage[margin=1in]{geometry} % 1 inch margins
\PassOptionsToPackage{hyphens}{url}\usepackage{hyperref} % used for website linkes to be displayed correctly
\usepackage{etoolbox} % used for changing vertical space before chapter headers to 0
\usepackage{titlesec} % customize title of TOC, and modify sections/subsections/etc.
\usepackage{sectsty} % used for centering the chapters and text for chapter
\usepackage{tocloft} % adjust spacing in TOC. adjust chapter and titles to be not bold in TOC
\usepackage{amsmath} % insert text inside of math sections
\usepackage{amssymb} % math symbols like Naturals, Integers, etc.

%%%%%%%%%%%%%%%%%%%%%%%%%%%%%%%%%%%%%%%%%%%%%%%%%%%%%%%%%%%%%%%%%%
% DOCUMENT SETUP

% displays subsubsection to TOC and doc
\setcounter{secnumdepth}{3}
\setcounter{tocdepth}{3}

% create the subsubsection
\titleformat{\subsubsection}[runin]
{\normalfont\normalsize\bfseries}
{\thesubsubsection}
{0.6em}
{}

% change bibliography to references
\addto\captionsamerican{\renewcommand{\bibname}{References}}

%%%%%%%%%%%%%%%%%%%%%%%%%%%%%%%%%%%%%%%%%%%%%%%%%%%%%%%%%%%%%%%%%%
% FORMATTING FOR EACH SECTION

% remove vertical space before & after the TOC title
\setlength{\cftbeforetoctitleskip}{0em}
\setlength{\cftaftertoctitleskip}{0em}

% redefine makechapterhead to remove spacing and apply font styles
\makeatletter
\def\@makechapterhead#1{
  {\parindent \z@ \raggedright \normalfont
    \ifnum \c@secnumdepth >\m@ne
        \centering \bfseries \normalsize \@chapapp\space \thechapter % center and normal size font with bold typesetting
        \par\nobreak
        \vskip 20\p@
    \fi
    \interlinepenalty\@M
    \normalsize #1\par\nobreak % normal size font
    \vskip 40\p@
  }}
\def\@makeschapterhead#1{
  {\parindent \z@ \raggedright
    \centering % centered text
    \normalfont % normal sized font
    \interlinepenalty\@M
    \normalsize  #1\par\nobreak % normal sized font
    \vskip 40\p@
  }}
\makeatother

% set each section/subsection/subsubsection to normal sized font
\sectionfont{\normalsize}
\subsectionfont{\normalsize}
\subsubsectionfont{\normalsize}

% change TOC name, center horizontally, and make non bold
\addto\captionsamerican{
    \renewcommand{\contentsname}{\hfill Table of Contents \hfill}
    \renewcommand{\cfttoctitlefont}{\normalsize}
    \renewcommand{\cftaftertoctitle}{\hfill}
}

% change chapters to be non-bold in TOC
\renewcommand{\cftchapfont}{\normalfont}
\renewcommand{\cftchappagefont}{\normalfont}

% setup for definitions, theorems, etc.
\newtheorem{thm}{Theorem}
\newtheorem{definition}{Definition}

%%%%%%%%%%%%%%%%%%%%%%%%%%%%%%%%%%%%%%%%%%%%%%%%%%%%%%%%%%%%%%%%%%
% BEGIN DOCUMENT

\begin{document}
    % title page & begin roman numeral numbering
    \pagenumbering{roman}
    \begin{titlepage}
        \centering
        CALIFORNIA STATE UNIVERSITY, NORTHRIDGE\\
        [1.5in]
        %Title
        Survey of Imperative Style Turing Complete proof techniques\\
        and an application to prove Proteus Turing Complete\\
        [1.5in]
        A thesis submitted in partial fulfillment of the requirements\\
        for the degree of Master of Science in\\
        Computer Science\\
        [0.5in]
        By\\
        [0.25in]
        Isaiah Martinez\\
        \vspace{\fill}
        December 2024
    \end{titlepage}

    \newpage
    % unsigned Signature Page
    \setcounter{page}{2} % set page counter in roman numerals to ii

    \chapter*{}\label{chapter:SignPage}
    \addcontentsline{toc}{chapter}{Signature Page}
        
        
        %this horizontal spacing keeps changing so just make sure it looks good in the pdf
        %such that it matches the vertical line to be above the 'K' in Kyle
        \hspace{0.9in}
        The thesis of Isaiah Martinez is approved:
        \vspace{2in}

        \begin{center}
            \begin{tabular}{p{3in} p{1in}}
                \rule{3in}{0.4pt} & \rule{1in}{0.4pt}\\
                Maryam Jalali, PhD. & Date \\
            \end{tabular}
        \end{center}

        \begin{center}
            \begin{tabular}{p{3in} p{1in}}
                \rule{3in}{0.4pt} & \rule{1in}{0.4pt}\\
                John Noga, PhD. & Date \\
            \end{tabular}
        \end{center}

        \begin{center}
            \begin{tabular}{p{3in} p{1in}}
                \rule{3in}{0.4pt} & \rule{1in}{0.4pt}\\
                Kyle Dewey, PhD., Chair & Date \\
            \end{tabular}
        \end{center}

        \vspace{\fill}

        \begin{center}
            California State University, Northridge
        \end{center}

    \newpage

    % acknowledgements
    % \chapter*{\normalfont Acknowledgements}\label{chapter:Acknowledgements}
    % \addcontentsline{toc}{chapter}{Acknowledgements}
    % \vspace{3in}
    %     \begin{center}
    %         Lorem ipsum dolor sit amet, consectetur adipiscing elit.
    %         Vivamus lacinia odio vitae vestibulum vestibulum.
    %         Cras venenatis euismod malesuada.
    %         Maecenas vehicula felis quis eros auctor, sed efficitur erat suscipit.
    %         Curabitur vel lacus velit. Proin a lacus at arcu porttitor vehicula.
    %         Mauris non velit vel lectus tincidunt ullamcorper at id risus.
    %         Sed convallis sollicitudin purus a scelerisque.
    %         Phasellus faucibus purus at magna tempus, sit amet aliquet nulla cursus.
    %     \end{center}
    

    \newpage

    \tableofcontents

    \newpage

    % List of figures/tables/illustrations
    % Delete any of the following 3 "List of __" that have < 4 items or are not scattered throughout the text

    % \chapter*{\normalfont List of Figures}\label{chapter:ListFigures}
    % \addcontentsline{toc}{chapter}{List of Figures}
    % \begin{center}
    %     This list must reference the figure, page it appears, and subject matter.
    % \end{center}
    
    % \newpage

    % \chapter*{\normalfont List of Illustrations}\label{chapter:ListIllustrations}
    % \addcontentsline{toc}{chapter}{List of Illustrations}
    % \begin{center}
    %     This list must reference the illustration, page it appears, and subject matter.
    % \end{center}
    
    \newpage

    \chapter*{\normalfont Abstract}\label{chapter:Abstract}
\addcontentsline{toc}{chapter}{Abstract}
\begin{center}
    Survey of Imperative Style Turing Complete proof techniques\\
    and an application to prove Proteus Turing Complete\\
    \vspace{0.15in}
    By\\
    \vspace{0.15in}
    Isaiah Martinez\\
    \vspace{0.3in}
    Master of Science in Computer Science\\
    \vspace{0.15in}
\end{center}

%i copied part of the other abstract from one of the sample theses. I can reword if need be.

Programming languages are provably Turing Complete.
Proofs showing a given system is Turing Complete come in many varieties depending on the discipline and perspective of the system.
Proteus is a programming language developed for the Autonomy Research Center for STEAHM at California State University, Northridge in collaboration with the NASA Jet Propulsion Laboratory.
This thesis looks into different methods of demonstrating a given system is Turing Complete and applying them to Proteus.
After completing the proof, there is a reflection of the methods discussed and utilized.

    \newpage

    \pagenumbering{arabic} % start normal page counting
    \section{Introduction}
Background on the topic

\subsection{Turing Machines}

What is a TM?

\subsection{Turing Complete}

What does TC mean?

How is it different from TMs?

\subsubsection{Example of Turing Complete Systems}

Link some funny/interesting examples of TC that have been proven.
    \chapter{Different Approaches for Proofs to Demonstrate Turing Completeness}\label{chapter:ProofApproachesForTC}

\section{Overview}\label{sec:ProofOverview}

We will be exploring the different approaches to demonstrate TC for different systems.
I have outlined the approaches based on their respective discipline, increasing in abstraction.
With each discipline comes a more theoretical view and understanding of TMs and TC systems.
My intention is to add clarity on the logic for these proofs/techniques.
For example, in the Computer Engineering perspective, TM is created from its mechanical properties through the usage of logic gates.
This is vastly different compared to how Mathematicians show TC, which is through the use of Lambda calculus -- a model for representing mathematical logic.
All proofs are equivalent in goal, however.
These are not the only perspectives and types of proofs for showing Turing Completeness as well as TMs.
This is simply a survey into what TMs and Turing Completeness looks like across the disciplines.

\section{Computer Engineering}\label{sec:CE}

In this section, we will analyze what a TM looks like from a physical perspective.
This may seem contradictory because the TM is described as a theoeretical machine.
But in fact, the very computers that we use today are are capable of processing TC systems through the usage of programming languages.
This means that they are limited TMs, because they are bounded only in memory.
In this approach, we will look at the core components of Computer Engineering to create a TM.

\subsection{Logical Design of a TM}\label{subsec:TMLogicalDesign}

To define what a TM does, we must explore what it is capable of.
Recall Theorem \ref{thm:CTT} in section \ref{subsec:Church-Turing Thesis}, "Every effectively calculable function can be computed by a Turing Machine."
Every effectively calculable function, as Turing and Church understood, was any mathematical calculation.
This means that a TM must have some ability to perform any operation on numbers, such as the basic operations of addition, subtraction, multiplication, and division.
Furthermore, they must be capable of combining these together to form more complex operations such as exponential arithmetic.
Beyond the mathematical aspect, they must allow for logical processing \cite{ChemTM}.

\subsubsection{Architecture}\label{subsubsec:Arch}

Looking at modern day computer architecture, there are several components that work independently but operate concurrently.
It is based off of the Modified Harvard Structure which is a variation of the Harvard computer architecture and Von Neumann architecture.
It combines both approaches towards computer architecture to handle many tasks that were difficult to handle using one of either architecture.

The von Neumann Architecture which has a centralized CPU to handle tasks for the computer.
All processes are handled by the CPU directly.
It contains several parts inside for processing data.
Inside the CPU is an ALU with registers, as well as a Control Unit.
The ALU processes arithmetic and logical computation, with the assistance of registers to store data at each step.
The Control unit determines the commands to be given to the ALU and other parts of the computer.
There is an associated Memory Unit which is where the bulk of memory storage lies.
Outside of the CPU are the Input and Output devices.

\begin{figure}[htb]
    \centering
    \includegraphics[width=10cm]{Images/Von_Neumann_Architecture.png}
       \caption{Von Neumann architecture.}
           \label{fig:VonNeumannArch}
\end{figure}

Figure \ref{fig:VonNeumannArch} visually describes the architecture \cite{vonNeumannImg}.
The von Neumann architecture has several limitations, with one of the biggest criticisms being that it is bottlenecked by the throughput between the CPU and memory.
Essentially, the CPU will eventually have more processing power than the bus can handle to write/read from memory.
This causes the CPU to wait until the bus is freed to continue processing.
As an alternative, we will now look at the Harvard Architecture

The Harvard architecture looks at the problem as says that if the CPU is too large and complex, then each individual component should be separated.
This allows for the tasks to be distributed evenly amongst the several smaller components like the ALU and Instruction memory as opposed to having them live inside the CPU.
The CPU is capable of simultaneous reads and writes.
However, a similar bottleneck occurs where the bus connecting each of the components is the limiting factor.
Figure \ref{fig:HarvardArch} illustrates the Harvard Architecture \cite{HarvardArchImg}.

\begin{figure}[htb]
    \centering
    \includegraphics[width=10cm]{Images/Harvard_architecture.svg.png}
       \caption{Harvard architecture.}
           \label{fig:HarvardArch}
\end{figure}

However, modern day computers utilize a mixed computer architecture called the Modified Harvard architecture.
It combines both architectures into a single model.
This usually is of the form of separating the components of the computer, but allowing several smaller memory caches for the CPU.
This is why modern computers utilize several components such as the CPU, GPU, Main Memory (SDD or HDD) and so forth.
Furthermore, this advancement allows for paralellism or multi-core systems to arise.
Some notable examples include the NVIDIA RTX 4080 which has over 8000 cores or for the intel i9-13900ks CPU to have 16 cores \cite{4080Specs,IntelSpecs}.
By allowing each one to independently operate, but still have the CPU as the "brains" of the operation.
We will now dive slightly deeper into the discussion to see what lies beneath these components within modern computer systems.

\subsubsection{Logic Gates}\label{subsec:LogicGates}

The basic building blocks for devices such as the ALU, CPU, and such are logic gates.
These are simplistic logical components that allow for processing of data and performing operations on them.

The core of the CPU relies on the ALU.
The ALU is contains registers which simply hold data inside.
Registers are associated with an address for referencing purposes.

Within the ALU there are smallers components that perform specific operations such as addition and subtraction.
These smaller components utilize logical gates such as the OR gate to compute the result with the given input.
See Figure \ref{fig:2BitALU} for a 2-Bit ALU that can process OR, AND, XOR, and addition calculations \cite{ALUImg}.

\begin{figure}[htb]
    \centering
    \includegraphics[width=10cm]{Images/2-bit_ALU.svg.png}
       \caption{A 2-Bit ALU.}
           \label{fig:2BitALU}
\end{figure}

EXPLAIN THE IMG
find a diff image for an ALU
    bitwise and, bitwise or
    addition
    subtraction

\subsection{Constructing the TM}\label{subsec:CreateTM}

With the ability to construct logic gates, we can create more complex components such as the ALU, CPU, and more.
Accompanied with the ability to store memory, as well as have a way to interact with the system through Input and Output, we are able to create a functional computer.
Developers have successfully created computers in the well-known video games of Minecraft and Terraria respectively in \cite{MCTM,TerrariaTM,TerrariaTMGH}.
In fact, we will see later on that the computer built inside Terraria is verifiably TC via a method discussed in \ref{subsubsec:CGoL}.

This means that to construct a TM on a physical level (of course alleviating the restriction of unbounded memory), these would be the minimum requirements \cite{nand2tetris,ELTCompSys}.

\section{Computer Science}\label{sec:CompSci}

In this section, we will conceptualize what a TM looks like under the lens of Computer Science.
There are 2 main perspectives: that of the Automata Theory and the Software Engineering approach.
The Automata Theory approach utilizes theoretical designs more reminiscent of those listed by Turing and Gavin.
The Software Engineering approach instead applies it to a problem to showcase Turing Completeness via programs and code.

\subsection{Automata Theory}\label{subsec:AutomataThy}

We will now abstract from the physical understanding of how to create a TM, to creating a theoretical one using Automata theory.
In automata theory, Turing Machines are described using logical notation.
The definition of a TM has several interpretations, but I will outline a slightly more advanced description \cite{IntroFormLangAuto,TuBB}.
The illustrated TM allows for the machine to stay at the current cell.

\begin{definition}
    A Turing Machine $M$ is defined by:
        \[M = (Q, \Sigma ,\Gamma, \delta, q_{0}, \raisebox{0.1cm}{\fbox{}}, F)\]
        \par \hangindent=3cm \hangafter=1
        where: \\
        \( Q \) is the set of internal states,\\
        \( \Sigma \) is the input alphabet,\\
        \( \Gamma \) is the finite set of symbols called the tape alphabet,\\
        \( \delta \) is the transition function,\\
        \( \raisebox{0.1cm}{\fbox{}} \in \Gamma \) is a special symbol called the blank,\\
        \( q_{0} \in Q \) is the initial state,\\
        \( F \subseteq Q \) is the set of final states.
\end{definition}

The transition function $\delta$ is defined as \[\delta: Q \times \Gamma \rightarrow Q \times \Gamma \times \{L, R, S\}.\]
This means that for a given $\delta$ transition with inputs $q \in Q$ and $a \in \Gamma$, the tape will move to another state $x \in Q$, clear the current cell (indicated by \raisebox{0.1cm}{\fbox{}}) or some symbol $y \in \Gamma$, and choose to move the tape head Left one cell (L), Right one cell (R), or to Stay at the current cell (S).
An example transition can be written: \[\delta(q_{0}, a) = (q_{1}, d, R)\] where the internal state is $q_{0}$, and we read input token a.
After the transition, we have internal state $q_{1}$, wrote symbol d onto the tape, and moved to the right one cell.
See Figure \ref{fig:DeltaTransition} demonstrating this change:

\begin{figure}[htb]
    \centering
    \includegraphics[width=16cm]{Images/deltatransition.png}
       \caption{Delta transition example from \cite{IntroFormLangAuto}.}
           \label{fig:DeltaTransition}
\end{figure}

Recall Figure \ref{fig:TM} which represents a simplistic TM.
In formal nomenclature, it can be written as follows:
\[
    \begin{aligned}
        Q &= \{q_{0}, q_{1}, q_{2}\} \text{ with associated labels \{Even, Odd, Halt\}}\\
        \Sigma &= {0,1}\\
        \Gamma &= {0,1}\\
        F &= \{q_{2}\}\\
        q_{0} &\in \text{ Q as the initial state}
    \end{aligned}
\]
and
\[
    \begin{aligned}
        \delta (q_{0}, 0) &= (q_{2}, 1, S),\\
        \delta (q_{0}, 1) &= (q_{1}, \raisebox{0.1cm}{\fbox{}}, R),\\
        \delta (q_{1}, 0) &= (q_{2}, 0, S),\\
        \delta (q_{1}, 1) &= (q_{0}, \raisebox{0.1cm}{\fbox{}}, R).\\
    \end{aligned}
\]

\subsubsection{Notable examples using Formal language}\label{subsubsec:NotableEgsFormalLang}

The proofs constructed using formal language usually modify the given system to meet these requirements.

In "Magic: the Gathering is Turing Complete", the authors modified the way the game is understood between 2 players.
They make the system force moves through clever leverage of the cards and their functions within the game \cite{MtGTC}.

In a different paper, "Turing Completeness and Sid Meier's Civilization", the system was also creatively modified to demonstrate Turing Completeness.
In each game, they constructed UTMs by utilizing the layout of the maps, as well as mechanics for changing states of the roads within the game \cite{CivTC}.

To show that Java Generics are TC, the authors showed that by creating a subtyping machine, it corresponds to only a small portion of the Java Generics while simulating TMs.
Following this discovery, they simulate a TM and then show that the given inputs are undecidable \cite{JavaGenericsTC}; leveraging an extension of Rice's Theorem, Theorem \ref{thm:RiceThm}.

The idea of describing the system as a TM and showing it has an undecidable input is a common practice.
This technique is also seen in a paper titled "The Game Description Language is Turing Complete" \cite{GDLTC}.

\subsection{Software Implementation}\label{subsec:SoftwareImplementation}

As opposed to the various theoretical approaches seen previously in section \ref{subsec:AutomataThy}, this section outlines a different perspective.
Instead of constructing a TM within the compounds of the system, an equivalent proof is to implement a program that demonstrates TC.
By implementing any known TC program successfully implies that the overall system is TC.

One such implementation would be to create a functional example of a known TC cellular automata.
Cellular automata are models of computation which use grids of cells.
Each cell contains a finite number of states, belonging to only one at any given time.
There are rules that determine what state a cell should become.
These rules are applied to all cells simultaneously, and thus form the next step in the sequence.
These steps are made sequentially to show the changes over time.
This makes all cellular automata 0-player games, meaning after an initial configuration there is no further input from the user.
With the work from Stephen Wolfram and other researchers such as Matthew Cook, some of these rules of cellular automata have been shown to be TC.
Famous examples of cellular automata include Conway's Game of Life and Rule 110 \cite{CellAutWiki,CellAutWolfram}.

Cellular automata are sorted into 4 classes:
\begin{itemize}
    \item Class 1: Nearly all initial patterns evolve quickly into a stable, homogenous state.
    Any randomness in the initial pattern disappears.
    \item Class 2: Nearly all initial patterns evolve quickly into stable or oscillating structures.
    Some of the randomness in the initial pattern may filter out, but some remains.
    Local changes to the initial pattern tend to remain local.
    \item Class 3: Nearly all initial patterns evolve in a pseudo-random or chaotic manner.
    Any stable structures that appear are quickly destroyed by the surrounding noise.
    Local changes to the initial pattern tend to spread infinitely.
    \item Class 4: Nearly all initial patterns evolve into structures that interact in complex and interesting ways, with the formation of local structures that are able to survive for long periods of time.
    Class 2 type stable or oscillating structures may be the eventual outcome, but the number of required to reach this state may be very large, even when the initial pattern is realtively simple.
    Local changes to the initial pattern may spread indefinitely.
\end{itemize}

Wolfram conjectured that many class 4 cellular automata are capable of universal computation.
Both Conway's Game of Life and Rule 110 exhibit "Class 4 behavior" and have been proven to be Turing Complete \cite{CGoLTM,CellAutBook}.

\subsubsection{Conway's Game of Life}\label{subsubsec:CGoL}

Conway's Game of Life is a 2D grid of cells extending infinitely in the cartesian plane.
Each cell may switch between only 2 states: Alive (On) or Dead (Off).
The rules of CGoL are simple and are illustrated in Figure \ref{fig:cgolrules} \cite{CGoLImg}.
CGoL is considered undecidable.
This is because given any initial pattern and a desired pattern at some later generation, there is no algorithm to determine whether the desired pattern will exist.
As such, it is analogous to the Halting Problem.

\begin{enumerate}
    \item Any live cell with fewer than two live neighbors dies. (Underpopulation)
    \item Any live cell with two or three live neighbors lives on to the next generation. (Survival)
    \item Any live cell with more than three live neighbors dies. (Overpopulation)
    \item Any dead cell with exactly three live neighbors becomes a live cell. (Reproduction)
\end{enumerate}

\begin{figure}[h!]
    \centering
    \includegraphics[width=10cm]{Images/CGoL.png}
       \caption{Rules of Conway's Game of Life visualized.}
            \label{fig:cgolrules}
\end{figure}

\newpage

\subsubsection{Rule 110}\label{subsubsec:Rule110}

Whereas CGoL is created on a 2D plane, Rule 110 lives in the 1D space.
There is an infinite tape of cells that each may exist in one of two states: 0 or 1.
By looking at three cells in series, one can find what the next state of the middle cell will be.
Below are the rules for Rule 110 with an associated graphic in Figure \ref{fig:Rule110} \cite{Rule110Img}:

\begin{enumerate}
    \item 111 makes 0
    \item 110 makes 1
    \item 101 makes 1
    \item 100 makes 0
    \item 011 makes 1
    \item 010 makes 1
    \item 001 makes 1
    \item 000 makes 0
\end{enumerate}

\begin{figure}[htb]
    \centering
    \includegraphics[width=10cm]{images/rule110.png}
       \caption{Rules for Rule 110}
           \label{fig:Rule110}
\end{figure}

Rule 110 is one of the simplest TC system that is known.
This makes it a relatively easy system to create to demonstrate Turing Completeness as opposed to CGoL.

\subsubsection{Programmable Calculator}\label{subsubsec:ProgCalc}

An entirely different approach to create a program that demonstrates Turing Completeness is to model the behavior of TMs directly.
This means that you create a system that does everything that a TM can do.
Recalling the Church-Turing Thesis (Theorem\ref{thm:CTT}), it must be able to calculate any function.
In a basic sense, this means that the system is capable of:

\begin{itemize}
    \item Reading/Writing memory
    \item Elementary Arithmetic/Logical operations
    \item Conditional Logic
    \item Looping Logic
\end{itemize}

Programmable Calculators meet all of these requirements.
By being able to store values into variables which can be referenced later, it can read/write memory.
Because it is a calculator, it is capable of performing arithmetic operations.
If statements and while loops are sufficient for handling the conditional and looping logic.
A programmable calculator thefore is TC \cite{CalcTC}.
This much simpler approach is clear and follows software engineering design principles.
A simplistic set of steps is outlined below:

\begin{enumerate}
    \item Start by making a basic arithmetic calculator.
    \item Then add the ability to store values into variables.
    \item Afterwards, create functionality for if statements, allowing boolean logic.
    \item Finally, create looping logic with while statements.
\end{enumerate}

\subsubsection{Interpreter for a known Turing Complete language}\label{subsec:InterpreterTC}

Alternatively, to show a programming language is TC, one can create an interpreter for a known TC language.
Many programming languages feature complex grammars and rulesets, which is why TC esoteric programming languages are preferred.
In fact brainfuck, as seen in section \ref{subsubsec:EsotericPL}, is used to demonstrate TC for its concise ruleset \cite{CBfInter,MeepWebsite,MeepGH,PythonBfInt}.

\section{Mathematics}\label{sec:Maths}

In this section, I will take a look at the mathematical system that is most well known for being TC, Lambda Calculus.
This is an abstract form of understanding functions and their capabilities.
It was actually designed by Church, and proven to be TC later on based off the work of Turing and Church by a famous mathematician: Stephen Cole Kleene \cite{LambdaCalcKleene}.

\subsection{Lambda Calculus}\label{subsec:LambdaCalc}

Lambda calculus upon initial inspection seems like a very abstract form of functions and relations within mathematics.
It can be understood to those in Computer Science as a very abstract programming language, and actually forms the basis of Functional Programming Languages \cite{TutLambdaCalc,FuncProgrChap}.

Lambda Calculus is a form of expressing functions in a simple manner that allows for creating any complex system \cite{LambdaCalcRG}.
At its core, it consists of three inductive rules defining what lambda terms are.
Each lambda term is a valid statement in lambda calculus:
\begin{enumerate}
    \item \textit{x}: A \textbf{variable} to represent a character or string.
    This is to be understood as a parameter for functions.
    \item \textit{$\lambda x.M$}: A lambda \textbf{abstraction} that is a function definition.
    This function takes the bound variable \textit{x} as input, and returns the body \textit{M}.
    \item \textit{(M N)}: An \textbf{application} where it applies the function \textit{M} to argument \textit{N}.
\end{enumerate}

There also exist reduction operations to improve legibility but retain equivalent logical meaning:
\begin{enumerate}
    \item \textit{($\lambda x.M[x]$) $\rightarrow$ ($\lambda y.M[y]$)}: $\alpha$-conversion, which renames the bound variables in the expression.
    This is be used to avoid name collisions.
    \item \textit{(($\lambda x.M$)\textit{N}) $\rightarrow$ ($M[x:=N]$)}: $\beta$-reduction, which replaces bound variables with the argument expression in the body of the abstraction.
    This is used to simplify chained functions being written out.
\end{enumerate}

Parentheses may be used to to disambiguate terms from each other.
This is especially useful when constructing complex applications using lambda calculus \cite{LambdaCalcWiki}.

I will define an equivalent TM to the previously mentioned TM seen in Figure \ref{fig:TM} and in section \ref{subsec:AutomataThy}.
Recall that the goal of the TM was to determine if there are an even or odd number of '1's in a sequence.

We construct the list of Natural Numbers, $\mathbb{N}$, as follows:
\[
    \begin{aligned}
        0 &\equiv \lambda sz.s(z)\\
        1 &\equiv \lambda sz.s(s(z))\\
        2 &\equiv \lambda sz.s(s(s(z)))\\
        &\text{and so on...}\\
    \end{aligned}
\]

Now we construct the ideas of Arithmetic Boolean Logic, and other necessary logical operators.
Treat 'f' as a function and variables as only locally defined to their respective operator \cite{LambdaFuncsList}.
Some notation can be interpreted as SKI combinator calculus \cite{SKICalcWiki}.

\[
    \begin{aligned}
        K &:= \lambda xy. x \equiv X(X (X X)) \equiv X' X' X'\\
        S &:= \lambda xyz. (x z) (y z) \equiv X (X (X (X X))) \equiv X K \equiv X' (X' X')\\
        I &:= \lambda x. x \equiv S K S \equiv S K K \equiv X X\\
        Y &:= \lambda g. (\lambda x. g (x x)) (\lambda x. g (x x))\\
        SUCC &:= \lambda nfx. f (n f x)\\
        PRED &:= \lambda n fx. n (\lambda gh. h(g f)) (\lambda u. x) (\lambda u. u)\\
            &\equiv \lambda n. n (\lambda gk. \hspace{0.1cm} ISZERO \hspace{0.1cm} (g \hspace{0.1cm} 1) k \hspace{0.1cm} (PLUS (g \hspace{0.1cm}k) 1))  (\lambda v. 0) 0\\
        PLUS &:= \lambda mnfx. n f (m f x)\\
            &\equiv \lambda mn. n \hspace{0.1cm} SUCC \hspace{0.1cm} m\\
        SUB &:= \lambda mn. n \hspace{0.1cm} PRED \hspace{0.1cm} m\\
    \end{aligned}
\]
\[
    \begin{aligned}
        MULT &:= \lambda mnf. m(n \hspace{0.1cm} f)\\
            &\equiv \lambda mn. m (PLUS \hspace{0.1cm} n) 0\\
        DIV &:= \lambda Y (\lambda gqab. \hspace{0.1cm} LT \hspace{0.1cm} a \hspace{0.1cm} b \hspace{0.1cm} (PAIR \hspace{0.1cm} q \hspace{0.1cm} a)(g \hspace{0.1cm} (SUCC \hspace{0.1cm} q) (SUB \hspace{0.1cm} a \hspace{0.1cm} b) \hspace{0.1cm} b)) 0\\
        MOD &:= \lambda ab. \hspace{0.1cm} CDR \hspace{0.1cm} (DIV \hspace{0.1cm} a \hspace{0.1cm} b)\\
        TRUE &:= \lambda xy. x \equiv K\\
        FALSE &:= \lambda xy. y \equiv 0 \equiv \lambda x. I \equiv K I \equiv S K \equiv X (X X)\\
        NOT &:= \lambda pab. p  b a \equiv \lambda p. p \hspace{0.1cm} FALSE \hspace{0.1cm} TRUE\\
        ISZERO &:= \lambda n. n (\lambda x. \hspace{0.1cm} FALSE) \hspace{0.1cm} TRUE\\
        LT &:= \lambda ab. \hspace{0.1cm} NOT \hspace{0.1cm} (LEQ \hspace{0.1cm} b \hspace{0.1cm} a)\\
        LEQ &:= \lambda mn. \hspace{0.1cm} ISZERO \hspace{0.1cm} (SUB \hspace{0.1cm} n \hspace{0.1cm} m)\\
        PAIR &:= \lambda xyf. f x y\\
        CAR &:= \lambda p. p \hspace{0.1cm} TRUE\\
        CDR &:= \lambda p. p \hspace{0.1cm} FALSE\\
        NIL &:= \lambda x. \hspace{0.1cm} TRUE\\
        NULL &:= \lambda p. p (\lambda xy. \hspace{0.1cm} FALSE)\\
        LENGTH &:= Y \lambda (gcx. \hspace{0.1cm} NULL \hspace{0.1cm} x c (g \hspace{0.1cm} (SUCC \hspace{0.1cm} c) \hspace{0.1cm} (CDR \hspace{0.1cm} x))) 0\\
    \end{aligned}
\]

Now one can combine these lambda functions from a higher abstraction level to perform the operation.

\begin{verbatim}
    Obtain the Length of the List.
    With the list length, subtract 1 from it.
    Take the mod of the result.
    If the new result is 0, then that means it was even.
    If instead it was 1, then it was odd.
\end{verbatim}

Resulting in the following simplified lambda calculus operation:\[MOD \hspace{0.1cm} (SUB \hspace{0.1cm} (LENGTH \hspace{0.1cm} (\textbf{input}) \hspace{0.1cm} 1)) \hspace{0.1cm} 2\]
with \textbf{input} being the input string.
One can expand this result to the above lambda calculus notation, resulting in an extraneously long sequence.
See Figure \ref{fig:OddEvenLambda} for an example of an expanded lambda calculus function that determines if a number is even or odd, i.e. it's cardinality \cite{RedditLambdaCalcPost,RedditLambdaCalcComment}.

\begin{figure}[htb]
    \centering
    \includegraphics[width=16cm]{Images/oddevenlambda.png}
       \caption{Expanded lambda calculus function to determine the cardinality of a number.}
           \label{fig:OddEvenLambda}
\end{figure}

With the ability to define any calculable function, Lambda Calculus is TC, as stated in the Church-Turing Thesis, Theorem \ref{thm:CTT}.
    \chapter{Proteus is Turing Complete}\label{chapter:ProteusTC}

This section will describe how we will construct the proof showing that Proteus is TC.

\section{Useful information to be used in the proof}

First, I will discuss features about Proteus programs on a theoretical level.
Then, I will discuss features about the Proteus language that allow for the creation of a TM.
This is the background information that will guide the construction of the proof outline and ultimately the proof itself.

\subsection{Undecidable input}\label{subsec:UndecidableInput}

Because Proteus is a higher-level programming language, we can leverage the usage of Rice's Theorem, Theorem \ref{thm:RiceThm}.
Thus, given any input it is impossible to determine an answer to the Halting Problem.
Furthermore, one cannot determine if there is an actor that will be told to switch to a particular state.
With this knowledge, it is understood that any given Proteus program is undecidable.
Thus, I will look at how to create a TM in Proteus.

\subsection{Requirements of a TM}\label{subsec:ReqsofTM}

In this section, I will point out critical pieces of Proteus that prove useful to create a TM.
We can see that the core features to create a TM, seen previously in sections \ref{subsec:TMLogicalDesign} and \ref{subsubsec:ProgCalc}, include:
\begin{enumerate}
    \item Arithmetic and Logical Processing
    \item Memory storage and manipulation
    \item Conditional Logic
    \item Looping Logic
    \item Input/Output
\end{enumerate}

Recall the proteus grammar seen in section \ref{subsec:ProteusGrammar}.
I will now describe from the Proteus grammar how to construct/use Proteus creating each part of the TM.

\subsubsection{Arithmetic and Logical Processing}\label{subsubsec:ArithLogProc}

The grammar provides the following definitions for arithmetic and logical processing:
\begin{itemize}
    \item BinOp
    \item Type
    \item ConstExpr
\end{itemize}

'BinOp' handles all binary operations for both arithmetic and logical calculation.
Some features include addition, subtraction, multiplication, division, modular arithmetic, equivalence relations, and, and or.
Looking at brainfuck in section \ref{subsubsec:EsotericPL}, one can notice that the only necessary mathematical operations are addition and subtraction.
Furthermore, the only logical processing is seen in the looping mechanism.
If the value at the pointer is 0 and the input token is a '[', then the loop is skipped.
This means that there is an equality check which returns a boolean result.

Looking deeper at the types of Proteus, type consists of all the possible types that are built into the language:
\begin{itemize}
    \item int
    \item string
    \item boolean
    \item actorname
    \item statename
    \item eventname
\end{itemize}

Despite allowing for division, the set of integers is closed under truncation, which is how Proteus handles cases where normally it wouldn't be.
eg. 5 / 2 = 2.5, but under truncation 5 / 2 = 2.
These truncation rules are similar to those seen in other languages such as Java and C \cite{TruncJava,TruncC}.

'ConstExpr' describes the 3 simple data types: Int, String, and Boolean.
These 3 types are capable of mimicking the behavior of brainfuck as well.

\subsubsection{Memory Storage and Manipulation}\label{subsubsec:MemStoManip}

The grammar provides the following definitions for memory storage and manipulation:
\begin{itemize}
    \item DefHSM
    \item DefState
    \item DefGlobalConst
    \item DecStmt
    \item AssignStmt
    \item SendStmt
\end{itemize}

'HSM' are Hierarchical State Machines which are actors in the language.
These state machines utilize states to determine logical processing.
These logical processes may utilize local or global variables that are stored, via the 'DecStmt' and 'DefGlobalConst' definitions respectively.

To modify data, the 'AssignStmt' was defined which allows for modifying the value of a given variable.
State Machines can modify state via the 'SendStmt' command.
Utilizing 'SendStmt', state machines can modify the state of themselves and other state machines as well.

\subsubsection{Conditional Logic}\label{subsubsec:CondLog}

The grammar provides the following definitions for conditional logic:
\begin{itemize}
    \item GoStmt
    \item JustGoStmt
    \item GoIfStmt
    \item ElseGoStmt
    \item IfStmt
\end{itemize}

Conditional Logic or Branching is necessary for a TM to compute any calculable function (see: Theorem \ref{thm:CTT}).
'GoStmt' is considered either a 'JustGoStmt' or a 'GoIfStmt', which are used to switch between states of a given HSM.
Similarly, the 'ElseGoStmt' switches to a particular state of a given HSM if the condition from the 'GoIfStmt' fails.

The 'IfStmt' is utilized for conditional logic within the processing of the state machines, and is akin to the standard if statements in other programming languages.
It is defined recursively to allow for nested "If ... else if.... else ..." statements.
These definitions allow for conditional statements to occur for a given HSM and within the code itself.

\subsubsection{Looping Logic}\label{subsubsec:LoopLog}

The only looping logic that can be seen in the garmmar that is built in, is the:
\begin{itemize}
    \item WhileStmt
\end{itemize}

This is the only necessary form of looping, as it can be broken by conditional statements and is capable of performing like other loops such as the do-while, for, and so forth.
This allows for more complex logical processing, such as recursion, which is a necessary requirement for TMs to perform any calculation.
A simplistic example of a problem that requires recursion would be the Ackermann Function.
See the definition of the Ackermann function here:
\[
\begin{aligned}
    A(0, n) &= n + 1\\
    A(m + 1, 0) &= A(m,1)\\
    A(m + 1, n + 1) &= A(m,\hspace{0.1cm} A(m + 1,n))
\end{aligned}
\]

Although being able to compute the Ackermann function requires recursion, it doesn't conclude that any system that can compute it is TC.
It was created to show that not all total computable functions are primitively recursive \cite{AckermannPR}.
The Ackermann function exists to show that not all functions can be represented with for loops, which is what primitive recursive functions are \cite{RecursiveFuncs}.
Nonetheless, all computable functions (regardless of their expression) are capable of being calculated by a TM, as stated by the Church-Turing Thesis (Theorem \ref{thm:CTT}).

\subsubsection{Input/Output}\label{subsubsec:IO}

Looking at the grammar definitions for:
\begin{itemize}
    \item Stmt
    \item PrintlnStmt
    \item PrintStmt
    \item SendStmt
\end{itemize}

From 'Stmt' I would like to highlight the 'SendStmt' command.
'SendStmt' is utilized to send events to a particular State Machine (i.e. an output).
By default, all actors are able to receive events.
'PrintlnStmt' and 'PrintStmt' are the standard print and println commands that are well known from other languages which serve as output to the console.
Although there is no explicit way to allow for input from the systems grammar dynamically, this is unnecessary as it can be preconfigured before runtime.
Thus, there exists a way to send inputs before the program is run via static input of values.

\section{Proteus Turing Machine Description}\label{sec:ProteusTMDescr}

By showing that any input to Proteus programs are undecidable and it is possible to create a TM in Proteus, Proteus can be shown to be TC.
This proof leverages the usage of both the Church-Turing Thesis, Theorem \ref{thm:CTT}, and Rice's Theorem, Theorem \ref{thm:RiceThm}.

I will explicitly create a TM using the built-in features seen previously in section \ref{subsec:ReqsofTM}.
After showing how to create a TM within Proteus, I will use Proteus to implement Conway's Game of Life and Rule 110 .
This is to demonstrate that the system is TC.
Recall demonstrating an implementation of CGoL or Rule110 indicates the system is TC from sections \ref{subsubsec:CGoL} and \ref{subsubsec:Rule110}.

\begin{enumerate}
    \item Define the set of internal states
    \item Define the initial state
    \item Define the final state
    \item Define the input alphabet
    \item Define the tape alphabet
    \item Define the state transitions
    \item Define the blank symbol
\end{enumerate}

I will now describe how to create a TM within Proteus from a higher abstraction layer with section \ref{sec:DefnTMProteus} going into the formal definition.
The tape is a series of state machines, HSMs, that will be ordered as $c_{0}, c_{1}, \dots , c_{\text{n}}$ arbitrarily with $c_{\text{n}}$ being the last non-empty cell.
This order will be consistent and not allow state machines to swap places with each other in the sequence.
There will be an additional state machine which functions as the read/write head.
The read/write head will be the one describing what the state of the TM and overall program is.
It contains a queue of events to be broadcast, with each entry in the queue containing a single event and target state machine.
Whenever a non-empty cell is encountered by the read/write head, it will broadcast what the current state is.
Figure \ref{fig:ProteusTMDesign} shows the design of the TM in a simplified manner.

\begin{figure}[h!]
    \centering
    \includegraphics[width=16cm]{images/ProteusTMDesign.png}
       \caption{Design of TM in Proteus}
           \label{fig:ProteusTMDesign}
\end{figure}

When the Proteus program is run, the read/write head will enter the 'ProgramOn' State.
If the tape is empty, as in there are no state machines that are created by the programmer, then the read/write head enters the 'ProgramOff' state and halts.
If instead the cell is non-empty, then the read/write head enters the 'Read' state which begins the process of reading information from the tape.
If a write is to be issued, then the read/write head enter the 'Write' state and writes the new data in the current cell.
After the write, the read/write enters the 'Read' state once again.

The logic for movement to an adjacent cell is mirrored on the left and right sides.
I will describe the movement to the left-adjacent cell.
From the current cell, the read/write head enters the 'BoundLeft' state and determines whether it encounters another symbol or the blank.
If there is a symbol, then it still lies within the non-empty tape information and returns to the 'Read' state.
If instead there is a blank, this means that there is no state machine defined, and has extended past the bounds of the tape with given information.
The read/write head moves one cell to the right, then returns to the 'Read' state to continue processing.
Because Proteus does not allow for dynamic state machine creation, the read/write head leaves the blank unmodified along the tape.

To exit the program and enter the halting state, all state machines within the tape must enter the 'Off' state, indicated by the 'O' within Figure \ref{fig:ProteusStateTM}.
The read/write head enters the 'FindStart' state from the 'Read' state to prepare for halting.
In the 'FindStart' state, the read/write head will move to the left, one cell at a time.
When it encounters a blank cell, it moves the read/write head to the adjacent right cell and enters the 'Check' state.
In the check state, the read/write head moves one cell at a time to the right and checks if they are in the 'Off' state.
Upon encountering a cell that is not in the 'Off' state the read/write head enters the 'Read' state for further processing.
If every cell is in the 'Off' state, then the read/write head will encounter a blank on the next cell after $c_{\text{n}}$.
In this case, all state machines are in the 'Off' state and the read/write head enters the 'ProgramOff' state to halt.

Because the read/write head is itself a state machine, it contains an event queue for events to be broadcasted.
Each event is associated to a single state machine.
In order to find the proper state machine to send the event to, the read/write head must search for it within the bounds of the non-empty tape.
This searching cannot utilize 'FindStart', because if all state machines are off and there are some nonzero number of events still in the queue, then the machine will still have events to process, but end up in the 'ProgramOff' state.
As such, it must search for them and find them using an unoptimized algorithm such as brute forcing all possible movements across the non-empty tape.
Note that I will explicitly describe what values $x, y, \text{O}$ must be in the following section \ref{sec:DefnTMProteus}.

\begin{figure}[h!]
    \centering
    \includegraphics[width=16cm]{images/ProteusTM.png}
       \caption{TM State Diagram in Proteus}
           \label{fig:ProteusStateTM}
\end{figure}

\section{Definition of a Turing Machine in Proteus}\label{sec:DefnTMProteus}

The list of internal states describes the states that the read/write head has.
The list of states is seen in Figure \ref{fig:ProteusStateTM} and will be described formally below.
\[
Q = \{\text{'ProgramOn'}, \text{'ProgramOff'}, \text{'Read'}, \text{'Write'}, \text{'BoundLeft'}, \text{'BoundRight'}, \text{'FindStart'}, \text{'Check'}\}
\]

The initial state of the program is 'ProgramOn', thus \[q_{0} = \text{'ProgramOn'}.\]

There is only a single halting state, 'ProgramOff'.
Therefore, the set of Final states is written, \[F = \{\text{'ProgramOff'}\}.\]

The input alphabet consists of the symbols that appear as already existing on the tape.
Recall that each cell is a state machine in Proteus, seen in section \ref{sec:ProteusTMDescr}.
The starting states for each cell (state machine) will be one of the following states: 'On' or 'Off'.
With this information we have the input alphabet:

\[
\Sigma = \{\text{'On'}, \text{'Off'}\}
\]

The symbols that can be written to and from the tape consist of the states within each state machine.
These are user defined, but also include the previously defined states: 'On' and 'Off'.
I will assume there is some number of states $n \in \mathbb{Z}_{\geq \text{0}}$ indicating that there exists a non-negative number of states defined by the programmer.
Each programmer created state, $s_i$, is not to be either of the 'On' or 'Off' states.
Because $\Gamma$ must contain the '$\raisebox{0.1cm}{\fbox{}}$' symbol, it indicates that it is capable of writing blanks to the cells.
Because Proteus disallows dynamic creation and deletion of state machines, I will say that writing a blank means the state machine at the specific cell is regarded as being in the 'Off' state.
Notice that in the described TM, there is no overwriting of data within the non-empty cells with a blank.
Because JFLAP does not allow for multiple characters to be a single element within the set of $\Gamma$, I use the symbol O to represent the 'Off' state in Figure \ref{fig:ProteusStateTM}.

\[
\Gamma = \{\text{'On'}, \text{'Off'}, s_{0}, \dots, s_{n}, \raisebox{0.1cm}{\fbox{}}\} \text{ for } n \in \mathbb{Z}_{\geq \text{0}}
\]

The transition function determines the conditions for the read/write head to change states.
These transitions can be seen clearly in Figure \ref{fig:ProteusStateTM}.

\[
    \begin{aligned}
        \text{let \hspace{0.1cm}} x, y \in \Gamma\\
        \delta (\text{'ProgramOn'}, x) &= (\text{'Read'}, x, S)\\
        \delta (\text{'ProgramOn'}, \raisebox{0.1cm}{\fbox{}}) &= (\text{'ProgramOff'}, \raisebox{0.1cm}{\fbox{}}, S)\\
%
        \delta (\text{'Read'}, x) &= (\text{'Write'}, x, S)\\
        \delta (\text{'Read'}, x) &= (\text{'BoundLeft'}, x, L)\\
        \delta (\text{'Read'}, x) &= (\text{'BoundRight'}, x, R)\\
        \delta (\text{'FindStart'}, x) &= (\text{'BoundLeft'}, x, S)\\
%
        \delta (\text{'Write'}, x) &= (\text{'BoundLeft'}, y, S)\\
%
        \delta (\text{'BoundLeft'}, x) &= (\text{'Read'}, x, S)\\
        \delta (\text{'BoundLeft'}, \raisebox{0.1cm}{\fbox{}}) &= (\text{'Read'}, \raisebox{0.1cm}{\fbox{}}, R)\\
%
        \delta (\text{'BoundRight'}, x) &= (\text{'Read'}, x, S)\\
        \delta (\text{'BoundRight'}, \raisebox{0.1cm}{\fbox{}}) &= (\text{'Read'}, \raisebox{0.1cm}{\fbox{}}, L)\\
%
        \delta (\text{'FindStart'}, x) &= (\text{'FindStart'}, x, L)\\
        \delta (\text{'FindStart'}, \raisebox{0.1cm}{\fbox{}}) &= (\text{'Check'}, \raisebox{0.1cm}{\fbox{}}, R)\\
%
        \delta (\text{'Check'}, x) &= (\text{'Read'}, x, S)\\
        \delta (\text{'Check'}, \text{'Off'}) &= (\text{'Check'}, \text{'Off'}, R)\\
        \delta (\text{'Check'}, \raisebox{0.1cm}{\fbox{}}) &= (\text{'ProgramOff'}, \raisebox{0.1cm}{\fbox{}}, S)\\
    \end{aligned}
\]

\newpage

In summary, the following definitions create a TM for an arbitrary Proteus program:
\[
    \begin{aligned}
        Q &= \{\text{'ProgramOn'}, \text{'ProgramOff'}, \text{'Read'}, \text{'Write'}, \text{'BoundLeft'}, \text{'BoundRight'}, \text{'FindStart'}, \text{'Check'}\}\\
        F &= \{\text{'ProgramOff'}\}\\
        q_{0} &= \text{'ProgramOn'}\\
        \Sigma &= \{\text{'On'}, \text{'Off'}\}\\
        \Gamma &= \{\text{'On'}, \text{'Off'}, s_{0}, \dots, s_{n}, \raisebox{0.1cm}{\fbox{}}\} \text{ for } n \in \mathbb{Z}_{\geq \text{0}}
    \end{aligned}
\]

with the transition functions:

\[
    \begin{aligned}
        \text{let \hspace{0.1cm}} x, y \in \Gamma\\
        \delta (\text{'ProgramOn'}, x) &= (\text{'Read'}, x, S)\\
        \delta (\text{'ProgramOn'}, \raisebox{0.1cm}{\fbox{}}) &= (\text{'ProgramOff'}, \raisebox{0.1cm}{\fbox{}}, S)\\
%
        \delta (\text{'Read'}, x) &= (\text{'Write'}, x, S)\\
        \delta (\text{'Read'}, x) &= (\text{'BoundLeft'}, x, L)\\
        \delta (\text{'Read'}, x) &= (\text{'BoundRight'}, x, R)\\
        \delta (\text{'FindStart'}, x) &= (\text{'BoundLeft'}, x, S)\\
%
        \delta (\text{'Write'}, x) &= (\text{'BoundLeft'}, y, S)\\
%
        \delta (\text{'BoundLeft'}, x) &= (\text{'Read'}, x, S)\\
        \delta (\text{'BoundLeft'}, \raisebox{0.1cm}{\fbox{}}) &= (\text{'Read'}, \raisebox{0.1cm}{\fbox{}}, R)\\
%
        \delta (\text{'BoundRight'}, x) &= (\text{'Read'}, x, S)\\
        \delta (\text{'BoundRight'}, \raisebox{0.1cm}{\fbox{}}) &= (\text{'Read'}, \raisebox{0.1cm}{\fbox{}}, L)\\
%
        \delta (\text{'FindStart'}, x) &= (\text{'FindStart'}, x, L)\\
        \delta (\text{'FindStart'}, \raisebox{0.1cm}{\fbox{}}) &= (\text{'Check'}, \raisebox{0.1cm}{\fbox{}}, R)\\
%
        \delta (\text{'Check'}, x) &= (\text{'Read'}, x, S)\\
        \delta (\text{'Check'}, \text{'Off'}) &= (\text{'Check'}, \text{'Off'}, R)\\
        \delta (\text{'Check'}, \raisebox{0.1cm}{\fbox{}}) &= (\text{'ProgramOff'}, \raisebox{0.1cm}{\fbox{}}, S)\\
    \end{aligned}
\]

% \section{Implementing Conway's Game of Life}\label{sec:ImplementCGoL}

% First I will describe the design for the overall program, then supplement it with the actual code implementation in Proteus.

% \subsection{Design of the Program}

% Recall CGoL in section \ref{subsubsec:CGoL}.
% To represent the standard 2-dimensional grid, I will use state machines for each cell.
% To remain consistent, I will be simulating the cartesian plane and assuming an arbitrary cell is the origin, (0,0).
% This is to allow for referencing the cells more digestible within the infinite plane of cells.

% To begin, let there be a non-zero amount of cells that are to be initialized to some starting state.
% I will define the bounds of the grid of initialized cells with $n,m \in \mathbb{N}$.
% Each cell will be labeled with it's respective position according to the origin, (0,0).
% Figure \ref{fig:ProteusCGoLDesign} illustrates the design of the grid with each of the cells.
% Note that I have only depicted the cells within the first quadrant because these are the cells where the initialization process is described \cite{CartesianPlane}.

% Also in Figure \ref{fig:ProteusCGoLDesign}, there is a 'Controller'.
% This is because all the cells within the plane will be state machines, alongside this 'Controller'.
% The 'Controller' does not exist within the plane described, and is to be viewed as a separate entity entirely.

% \begin{figure}[htb]
%     \centering
%     \includegraphics[width=10cm]{Images/CGoLDesign.png}
%        \caption{Design of System for Conway's Game of Life}
%            \label{fig:ProteusCGoLDesign}
% \end{figure}

% There will be two types of state machines: The cell, and the controller.
% I will now describe the data members associated with each state machine.

% The cell has 5 local variables:
% \begin{enumerate}
%     \item myName: the name of the cell as a string. (Final)
%     \item Xcoord: the x coordinate as an integer. (Final)
%     \item Ycoord: the y coordinate as an integer. (Final)
%     \item currCellIsOn: holds the value of the current state of the cell (true == On).
%     \item nextStateCurrCellIsOn: holds the value of the next state of the cell (true == On).
% \end{enumerate}

% Each cell also has 3 states: 
% \begin{enumerate}
%     \item Display: broadcasts currState to whatever machine requested the current state of the cell.
%     Is the initial state the cell is in.
%     \item CalculateNext: Asks the neighbors in the 4 cardinal directions for their Display.
%     Then determines based on the rules of CGoL what the next state of this cell should be.
%     \item Update: updates currState to the value of nextStateCurrCellIsOn
% \end{enumerate}

% The controller is much simpler.
% It has no local variables and only 2 states:

% \begin{enumerate}
%     \item Setup: Determines cells to turn on/off when starting the program
%     \item nextStep: broadcasts a message to all Cells to (1) calculate their next state and (2) update after finding their next state.
% \end{enumerate}

% \subsection{Implementation of the Conway's Game of Life}

% First, I will describe the events in Proteus code.
% In total there are _:
% \begin{enumerate}
%     \item getDisplay(myName): the current cell requests another cell for their currState value.
%     It passes the current cells name so that the receiver may transmit the data back to this cell.
%     \item calculateNextState(): this event is sent from the controller to all cells.
%     It instructs the cells to calculate their next state (true == On).
%     \item updateAllCells(): this event is sent from the controller to all cells.
%     It instructs the cells to update their currState to the value of nextStateCurrCellIsOn.
%     \item initializeCell(Value): this event is sent from the controller to all programmer-defined cells.
%     It instructs the cell to update it's currState to the given Value, with Value $in$ {true,false}.
% \end{enumerate}

% The proteus code for a given cell at coordinates (X,Y) are as follows:

% \begin{verbatim}
% actor cellXY{
%     string myName = "cellXY";
%     int Xcoord = X;
%     int Ycoord = Y;
%     bool currCellIsOn = false;
%     bool nextStateCurrCellIsOn = false;
%     statemachine {
%         initial Display;
%         state Display {
%             on getDisplay {otherCellName} {otherCellName ! currCellIsOn}
%             on getcalculateNextState {} {go calculateNext {}}
%             on updateAllCells {} {go Update {}}
%         }
%         state calculateNext {
%             int neighborTop = Ycoord + 1;
%             int neighborBot = Ycoord - 1;
%             int neighborLeft = Xcoord - 1;
%             int neighborRight = Xcoord + 1;
%             string neighborTopName = "cell" + Xcoord + neighborTop;
%             string neighborBotName = "cell" + Xcoord + neighborBot;
%             string neighborLeftName = "cell" + neighborLeft + Ycoord;
%             string neighborRightName = "cell" + neighborRight + Ycoord;
%             int count = 0;
%             if (neighborTopName ! getDisplay {myName}) {
%                 count += 1;
%             }
%             if (neighborBotName ! getDisplay {myName}) {
%                 count += 1;
%             }
%             if (neighborLeftName ! getDisplay {myName}) {
%                 count += 1
%             }
%             if (neighborRightName ! getDisplay {myName}) {
%                 count += 1
%             }
%             if ((!(currCellIsOn)) && (count == 3)) {
%                 nextStateCurrCellIsOn = true;
%             } else if ((currCellIsOn) && ((count == 2) || (count == 3))) {
%                 nextStateCurrCellIsOn = true;
%             } else if ((currCellIsOn) && (count < 2)) {
%                 nextStateCurrCellIsOn = false;
%             } else if ((currCellIsOn) ** (count > 3)) {
%                 nextStateCurrCellIsOn = false;
%             }
%             go Display {}
%         }
%         state Update {
%             currCellIsOn = nextStateCurrCellIsOn;
%             go Display {}
%         }
%     }
% }
% \end{verbatim}

% \section{Implementing Rule 110}\label{sec:ImplementRule110}

% Implementation of Rule 110 demonstrating Proteus is TC

% \subsection{Design of the Program}

% \subsection{Implementation of the Rule 110}

    \chapter{Conclusion}\label{chatper:Concl}

Succinctly describe what techniques were used.
Compare and constrast them? Perhaps a discussion section?

    \newpage

    % bibliography
    
    %alphabetical order
    %retrieval date is only for non-journal instances where material might change at a later date

    \begin{singlespace}

        \begin{thebibliography}{}

            \bibitem{NDTMeqDTM} Q. Gao and X. Xu, “The Analysis and Research on Computational Complexity,” pp. 3467–3472.

            \bibitem{GandyPaper} R. Gandy, “Church’s Thesis and Principles for Mechanisms,” The Kleene Symposium, pp. 123–148, Jun. 1980.

            \bibitem{ProteusRunTime} B. McClelland, “Adding Runtime Verification to the Proteus Language,” CSUN, May 2021.

            \bibitem{IntroFormLangAuto} P. Linz, An Introduction to Formal Languages and Automata. Jones \& Bartlett Learning, 2016.

            \bibitem{JavaGenericsTC} R. Grigore, “Java generics are turing complete,” ACM SIGPLAN Notices, vol. 52, no. 1, pp. 73–85, Jan. 2017, doi: https://doi.org/10.1145/3093333.3009871.

            \bibitem{MtGTC} A. Churchill, S. Biderman, and A. Herrick, “Magic: The Gathering is Turing Complete.”

            \bibitem{CivTC} A. de Wynter, “Turing Completeness and Sid Meier’s Civilization,” IEEE Transactions on Games, vol. 15, no. 2, pp.292-299, June 2023.

            \bibitem{GDLTC} A. Saffidine, “The Game Description Language Is Turing Complete,” IEEE Transactions on Computational Intelligence and AI in Games, vol. 6, no. 4, pp. 320–324, Dec. 2014, doi: https://doi.org/10.1109/tciaig.2014.2354417.

            \bibitem{CGoLTM} P. Rendell, "A Universal Turing Machine in Conway's Game of Life," 2011 International Conference on High Performance Computing \& Simulation, Istanbul, Turkey, 2011, pp. 764-772, doi: 10.1109/HPCSim.2011.5999906.
        
            \bibitem{TTTTM} S. S. T. Gontumukkala, Y. S. V. Godavarthi, B. R. R. T. Gonugunta and S. M., "Implementation of Tic Tac Toe Game using Multi-Tape Turing Machine," 2022 International Conference on Computational Intelligence and Sustainable Engineering Solutions (CISES), Greater Noida, India, 2022, pp. 381-386, doi: 10.1109/CISES54857.2022.9844404.

            \bibitem{ShallityAutomataThy} Jeffrey Outlaw Shallit, A second course in formal languages and automata theory. Cambridge ; New York: Cambridge University Press, 2009.

            \bibitem{JFLAPGrading} M. Biçer, F. Albayrak and U. Orhan, "Automatic Automata Grading System Using JFLAP," 2023 Innovations in Intelligent Systems and Applications Conference (ASYU), Sivas, Turkiye, 2023, pp. 1-4, doi: 10.1109/ASYU58738.2023.10296744.

            \bibitem{TuBB} E. Luce and S. H. Rodger, "A visual programming environment for Turing machines," Proceedings 1993 IEEE Symposium on Visual Languages, Bergen, Norway, 1993, pp. 231-236, doi: 10.1109/VL.1993.269602.
            
            \bibitem{LambdaCalcKleene} Dezani‐Ciancaglini Mariangiola and J. R. Hindley, “Lambda-Calculus,” Wiley Encyclopedia of Computer Science and Engineering, pp. 1–8, Sep. 2008, doi: https://doi.org/10.1002/9780470050118.ecse212.

            \bibitem{LambdaCalcRG} M. Dezani-Ciancaglini and J. R. Hindley, “Lambda-Calculus,” Nov. 2007, Available: \href{https://www.researchgate.net/publication/228107078_Lambda-Calculus}{https://www.researchgate.net/publication/228107078\_Lambda-Calculus}

            \bibitem{TutLambdaCalc} R. Rojas, “A Tutorial Introduction to the Lambda Calculus,” 2015, Available: \href{https://arxiv.org/pdf/1503.09060}{https://arxiv.org/pdf/1503.09060}

            \bibitem{AckermannPR} CWoo, “Ackermann function is not primitive recursive,” Mar. 2013, Available: \href{https://www.cs.tau.ac.il//~nachumd/term/42019.pdf}{https://www.cs.tau.ac.il//~nachumd/term/42019.pdf}
        
            \bibitem{RecursiveFuncs} W. Dean and A. Naibo, Recursive Functions, Fall 2024 Edition. Stanford University: Metaphysics Research Lab, Stanford University, 2024. Available: \href{https://plato.stanford.edu/archives/fall2024/entries/recursive-functions/}{https://plato.stanford.edu/archives/fall2024/entries/recursive-functions/}
        
            \bibitem{ChemTM} A. Hjelmfelt, E. D. Weinberger, and J. Ross, “Chemical implementation of neural networks and Turing machines.,” Proceedings of the National Academy of Sciences, vol. 88, no. 24, pp. 10983–10987, Dec. 1991, doi: https://doi.org/10.1073/pnas.88.24.10983.

            \bibitem{MCTM} “I Made a Working Computer with just Redstone!,” www.youtube.com. \href{https://www.youtube.com/watch?v=CW9N6kGbu2I}{https://www.youtube.com/watch?v=CW9N6kGbu2I}

            \bibitem{TerrariaTM} From Scratch, “I Made a 32-bit Computer Inside Terraria,” YouTube, Jun. 24, 2023. \href{https://www.youtube.com/watch?v=zXPiqk0-zDY}{https://www.youtube.com/watch?v=zXPiqk0-zDY}

            %This citation is weird because it wasn't being properly created from the ieee citation generator so i did it by hand. I hope it's okay
            \bibitem{TerrariaTMGH} misprit7, "computerraria", Feb. 16, 2023, GitHub repository, \href{https://github.com/misprit7/computerraria}{https://github.com/misprit7/computerraria}
            
            \bibitem{nand2tetris} “Home | nand2tetris,” nand2tetris, 2017. \href{https://www.nand2tetris.org/}{https://www.nand2tetris.org/}

            \bibitem{ELTCompSys} Noam Nisan, ELEMENTS OF COMPUTING SYSTEMS : building a modern computer from first principles. 2020.

            % The above sources are the actual journals and books read/used in the thesis.
            % the below sources are just citations for declarations of info which are not explicitly journals or formal publications

            \bibitem{WolframRiceThm} Sakharov, Alex. “Rice’s Theorem.” From MathWorld--A Wolfram Web Resource, created by Eric W. Weisstein, Wolfram.com, 2024. \href{https://mathworld.wolfram.com/RicesTheorem.html}{https://mathworld.wolfram.com/RicesTheorem.html}
            
            \bibitem{WikiRiceThm} Wikipedia Contributors, “Rice’s theorem,” Wikipedia, Sep. 24, 2019. \href{https://en.wikipedia.org/wiki/Rice%27s_theorem}{https://en.wikipedia.org/wiki/Rice\%27s\_theorem}
            
            \bibitem{BfWiki} “Brainfuck,” Wikipedia, Oct. 05, 2021. \href{https://en.wikipedia.org/wiki/Brainfuck}{https://en.wikipedia.org/wiki/Brainfuck}
            
            \bibitem{BfGH} 262588213843476, “Basics of BrainFuck,” Gist, Oct. 29, 2024. \href{https://gist.github.com/roachhd/dce54bec8ba55fb17d3a}{https://gist.github.com/roachhd/dce54bec8ba55fb17d3a}

            \bibitem{BfSO} speeder, “How does the Brainfuck Hello World actually work?,” Stack Overflow, May 30, 2013. \href{https://stackoverflow.com/questions/16836860/how-does-the-brainfuck-hello-world-actually-work/19869651#19869651}{https://stackoverflow.com/questions/16836860/how-does-the-brainfuck-hello-world-actually-work/19869651\#19869651}
                        
            \bibitem{SmallestBfCompiler} “brainfuck - Esolang,” Esolangs.org, 2023. \href{https://esolangs.org/wiki/Brainfuck#Self-interpreters}{https://esolangs.org/wiki/Brainfuck\#Self-interpreters}
            
            \bibitem{SmallestBfCompilerActual} Brainfuck.org, 2024. \href{https://brainfuck.org/dbfi.b}{https://brainfuck.org/dbfi.b}
            
            \bibitem{GitMadeinC} “git.git - The core git plumbing,” Kernel.org, 2024. \href{https://git.kernel.org/pub/scm/git/git.git/tree/}{https://git.kernel.org/pub/scm/git/git.git/tree/}

            \bibitem{LinuxMadeinC} “kernel/git/torvalds/linux.git - Linux kernel source tree,” Kernel.org, 2024. \href{https://git.kernel.org/pub/scm/linux/kernel/git/torvalds/linux.git/tree/}{https://git.kernel.org/pub/scm/linux/kernel/git/torvalds/linux.git/tree/}

            \bibitem{GNUPreferC}  “Source Language (GNU Coding Standards),” Gnu.org, 2024. \href{https://www.gnu.org/prep/standards/html_node/Source-Language.html}{https://www.gnu.org/prep/standards/html\_node/Source-Language.html}
            
            \bibitem{MarketShareOS} StatCounter, “Desktop Operating System Market Share Worldwide | StatCounter Global Stats,” StatCounter Global Stats, 2019. \href{https://gs.statcounter.com/os-market-share/desktop/worldwide/}{https://gs.statcounter.com/os-market-share/desktop/worldwide/}
            
            \bibitem{COBOLBanks} “On the past, present, and future of COBOL – Increment: Programming Languages,” increment.com. \href{https://increment.com/programming-languages/cobol-all-the-way-down/}{https://increment.com/programming-languages/cobol-all-the-way-down/}

            \bibitem{CSUNStudents} “How Does California State University--Northridge Rank Among America’s Best Colleges?,” @USNews, 2015. \href{https://www.usnews.com/best-colleges/california-state-university-northridge-1153}{https://www.usnews.com/best-colleges/california-state-university-northridge-1153}

            \bibitem{JSPopular} H. Shah, “7 Frontend JavaScript Frameworks Loved by Developers in 2022,” Simform - Product Engineering Company, Feb. 17, 2022. \href{https://www.simform.com/blog/javascript-frontend-frameworks/}{https://www.simform.com/blog/javascript-frontend-frameworks/}

            \bibitem{ChatGPTPython} D. K. K. has years of experience as a S. D. S. H. enjoys coding, teaching, and has created this website to make M. L. accessible to everyone, “Is ChatGPT Written In Python??? [We FINALLY Found The Proof]» EML,” enjoymachinelearning.com, Feb. 09, 2023. \href{https://enjoymachinelearning.com/blog/is-chatgpt-written-in-python/}{https://enjoymachinelearning.com/blog/is-chatgpt-written-in-python/}

            \bibitem{RDataSci} Simplilearn, “What is R: Overview, its Applications and what is R used for | Simplilearn,” Simplilearn.com, Oct. 25, 2021. \href{https://www.simplilearn.com/what-is-r-article}{https://www.simplilearn.com/what-is-r-article}

            \bibitem{DataSciLangs} J. C. Luna, “Learn R, Python \& Data Science Online,” www.datacamp.com, Mar. 2023. \href{https://www.datacamp.com/blog/top-programming-languages-for-data-scientists-in-2022}{https://www.datacamp.com/blog/top-programming-languages-for-data-scientists-in-2022}

            \bibitem{vonNeumannImg} Kapooht, “English: von Neumann Architecture,” Wikimedia Commons, Apr. 28, 2013. \href{https://commons.wikimedia.org/wiki/File:Von_Neumann_Architecture.svg}{https://commons.wikimedia.org/wiki/File:Von\_Neumann\_Architecture.svg}

            \bibitem{HarvardArchImg} “File:Harvard architecture.svg - Wikimedia Commons,” Wikimedia.org, May 11, 2010. \href{https://commons.wikimedia.org/wiki/File:Harvard_architecture.svg}{https://commons.wikimedia.org/wiki/File:Harvard\_architecture.svg}

            \bibitem{4080Specs} “NVIDIA GeForce RTX 4080 Specs,” TechPowerUp, Oct. 13, 2023. \href{https://www.techpowerup.com/gpu-specs/geforce-rtx-4080.c3888}{https://www.techpowerup.com/gpu-specs/geforce-rtx-4080.c3888}

            \bibitem{IntelSpecs} “Intel® CoreTM i9-13900KS Processor (36M Cache, up to 6.00 GHz) - Product Specifications,” Intel. \href{https://www.intel.com/content/www/us/en/products/sku/232167/intel-core-i913900ks-processor-36m-cache-up-to-6-00-ghz/specifications.html}{https://www.intel.com/content/www/us/en/products/sku/232167/intel-core-i913900ks-processor-36m-cache-up-to-6-00-ghz/specifications.html}

            \bibitem{ALUImg} “File:2-bit ALU.svg - Wikibooks, open books for an open world,” Wikibooks.org, Oct. 18, 2011. \href{https://en.wikibooks.org/wiki/File:2-bit_ALU.svg}{https://en.wikibooks.org/wiki/File:2-bit\_ALU.svg}

            \bibitem{CellAutWiki} Wikipedia Contributors, “Cellular automaton,” Wikipedia, Dec. 05, 2019. \href{https://en.wikipedia.org/wiki/Cellular_automaton}{https://en.wikipedia.org/wiki/Cellular\_automaton}

            \bibitem{CellAutWolfram} E. W. Weisstein, “Cellular Automaton,” From Mathworld--A Wolfram Web Resource, mathworld.wolfram.com. \href{https://mathworld.wolfram.com/CellularAutomaton.html}{https://mathworld.wolfram.com/CellularAutomaton.html}

            \bibitem{CellAutBook} A. Ilachinski, Cellular Automata. World Scientific, 2001.

            \bibitem{CGoLImg} “Math’s ‘Game of Life’ Reveals Long-Sought Repeating Patterns | Quanta Magazine,” Quanta Magazine, Jan. 18, 2024. \href{https://www.quantamagazine.org/maths-game-of-life-reveals-long-sought-repeating-patterns-20240118/}{https://www.quantamagazine.org/maths-game-of-life-reveals-long-sought-repeating-patterns-20240118/}

            \bibitem{Rule110Img} “File:One-d-cellular-automaton-rule-110.gif - Wikimedia Commons,” Wikimedia.org, Nov. 20, 2018. \href{https://commons.wikimedia.org/wiki/File:One-d-cellular-automaton-rule-110.gif}{https://commons.wikimedia.org/wiki/File:One-d-cellular-automaton-rule-110.gif}

            \bibitem{CalcTC} Code \& Optimism, “How to write a Turing-Complete Programming Language in 40 minutes in Ruby using Bable-Bridge,” YouTube, Sep. 19, 2012. \href{https://www.youtube.com/watch?v=_Uoyufkb5lk}{https://www.youtube.com/watch?v=\_Uoyufkb5lk}

            \bibitem{CBfInter} M. Kenyon, “How to Write a Brainfuck Interpreter in C\#,” Thesharperdev.com, Oct. 12, 2019. \href{https://thesharperdev.com/how-to-write-a-brainfuck-interpreter-in-c/}{https://thesharperdev.com/how-to-write-a-brainfuck-interpreter-in-c/}

            \bibitem{MeepWebsite} “Compiling to Brainf\#ck - Meep.,” InJuly, 2024. \href{https://injuly.in/blog/bfinbf/index.html}{https://injuly.in/blog/bfinbf/index.html}

            \bibitem{MeepGH} srijan-paul, “GitHub - srijan-paul/meep: A programming language that compiles to brainfuck.,” GitHub, 2020. \href{https://github.com/srijan-paul/meep}{https://github.com/srijan-paul/meep}

            \bibitem{PythonBfInt} M. Ueding, “Creating a Brainfuck interpreter,” Martin Ueding, Apr. 19, 2023. \href{https://martin-ueding.de/posts/creating-a-brainfuck-interpreter/}{https://martin-ueding.de/posts/creating-a-brainfuck-interpreter/}

            \bibitem{FuncProgrChap} “Functional Programming,” learn.saylor.org. \href{https://learn.saylor.org/mod/book/tool/print/index.php?id=33044&chapterid=13087}{https://learn.saylor.org/mod/book/tool/print/\-index.php?id=33044\&chapterid=13087}

            \bibitem{LambdaCalcWiki} Wikipedia Contributors, “Lambda calculus,” Wikipedia, Jan. 02, 2020. \href{https://en.wikipedia.org/wiki/Lambda_calculus}{https://en.wikipedia.org/wiki/Lambda\_calculus}

            \bibitem{LambdaFuncsList} “Collected Lambdas,” jwodder.freeshell.org. \href{https://jwodder.freeshell.org/lambda.html}{https://jwodder.freeshell.org/lambda.html}

            \bibitem{SKICalcWiki} Wikipedia Contributors, “SKI combinator calculus,” Wikipedia, Oct. 28, 2024. \href{https://en.wikipedia.org/wiki/SKI_combinator_calculus}{https://en.wikipedia.org/wiki/SKI\_combinator\_calculus}

            \bibitem{RedditLambdaCalcPost} “Reddit - Dive into anything,” Reddit.com, 2017. \href{https://www.reddit.com/r/ProgrammerHumor/comments/78z90f/when_you_need_to_know_if_a_number_is_even_or_odd/}{https://www.reddit.com/r/ProgrammerHu\-mor/comments/78z90f/when\_you\_need\_to\_know\_if\_a\_number\_is\_even\_or\_odd/} 

            \bibitem{RedditLambdaCalcComment} “Reddit - Dive into anything,” Reddit.com, 2017. \href{https://www.reddit.com/r/ProgrammerHumor/comments/78z90f/comment/doylzry/}{https://www.reddit.com/r/ProgrammerHu\-mor/comments/78z90f/comment/doylzry/}

            \bibitem{TruncJava} “What is truncation in Java - Javatpoint,” www.javatpoint.com, 2021. \href{https://www.javatpoint.com/what-is-truncation-in-java}{https://www.javatpoint.com/what-is-truncation-in-java}

            \bibitem{TruncC} GeeksforGeeks, “trunc() , truncf() , truncl() in C language,” GeeksforGeeks, Jan. 25, 2018. \href{https://www.geeksforgeeks.org/trunc-truncf-truncl-c-language/}{https://www.geeksforgeeks.org/trunc-truncf-truncl-c-language/}

            \bibitem{ExcelTC} “The Excel Formula Language Is Now Turing-Complete,” InfoQ. \href{https://www.infoq.com/articles/excel-lambda-turing-complete/}{https://www.infoq.com/articles/excel-lambda-turing-complete/}

            \bibitem{AccTC} “Accidentally Turing-Complete,” beza1e1.tuxen.de. \href{https://beza1e1.tuxen.de/articles/accidentally_turing_complete.html}{https://beza1e1.tuxen.de/articles/accident\-ally\_turing\_complete.html}

            \bibitem{AccTC2} G. Branwen, “Surprisingly Turing-Complete,” gwern.net, Dec. 2012, Available: \href{https://gwern.net/turing-complete}{https://gwern.net/turing-complete}

            \bibitem{SecVuln} “A deep dive into an NSO zero-click iMessage exploit: Remote Code Execution,” Blogspot.com, 2021. \href{https://googleprojectzero.blogspot.com/2021/12/a-deep-dive-into-nso-zero-click.html?m=1}{https://googleprojectzero.blogspot.com/2021/12/a-deep-dive-into-nso-zero-click.html?m=1}

            \bibitem{ActorModelParallel} Wikipedia Contributors, “Actor model,” Wikipedia, Nov. 14, 2019. \href{https://en.wikipedia.org/wiki/Actor_model}{https://en.wikipedia.org/wiki/Actor\_model}
           
            \bibitem{ActorJavaParallel} L. Nigro, “Parallel Theatre: An actor framework in Java for high performance computing,” Simulation Modelling Practice and Theory, vol. 106, p. 102189, Jan. 2021, doi: https://doi.org/10.1016/j.simpat.2020.102189. 

            \bibitem{CartesianPlane} “Cartesian Plane - Definition, Meaning, Quadrants, Examples,” Cuemath. https://www.cuemath.com/geometry/cartesian-plane/

            \bibitem{JavaUse1} “What Is Java Used For: 12 Real World Java Applications,” www.softwaretestinghelp.com. https://www.softwaretestinghelp.com/real-world-applications-of-java/

            \bibitem{JavaUse2} “What is Java Used For?,” Codecademy News, Jun. 24, 2021. https://www.codecademy.com/resources/blog/what-is-java-used-for/

            \bibitem{CPPUse1} Simplilearn, “Top 7 Practical Applications of C++ and the Way to Build a Career in the Field,” Simplilearn.com, Jan. 16, 2020. https://www.simplilearn.com/c-plus-plus-programming-for-beginners-article

            \bibitem{CPPUse2} “Uses of C++ | 10 Reasons Why You Should Use C++,” EDUCBA, Jul. 19, 2019. https://www.educba.com/uses-of-c-plus-plus/

            \bibitem{CPPFirefox} “mozilla/gecko-dev,” GitHub, Feb. 01, 2024. https://github.com/mozilla/gecko-dev

            \bibitem{FirefoxMain} “The new, fast browser for Mac, PC and Linux | Firefox,” Mozilla, 2000. https://www.mozilla.org/en-US/firefox/
        
            \bibitem{PHPScriptingServer} Astari S., “What Is PHP? Learning All About the Scripting Language,” Hostinger Tutorials, Apr. 30, 2019. https://www.hostinger.com/tutorials/what-is-php/

            \bibitem{PHPWP} http://facebook.com/syedbalkhi, “What is PHP? How PHP is Used in WordPress?,” WPBeginner, 2019. https://www.wpbeginner.com/glossary/php/

            \bibitem{PHPLaravel} T. Otwell, “Laravel - The PHP Framework For Web Artisans,” Laravel.com, 2015. https://laravel.com/

            \bibitem{TopLangs1} L. dev, “Top 8 Most Demanded Programming Languages in 2023,” Devjobsscanner, Jun. 22, 2023. https://www.devjobsscanner.com/blog/top-8-most-demanded-programming-languages/

            \bibitem{TopLangs2} L. Whitney, “Top 10 programming languages employers want in 2023,” TechRepublic, Feb. 03, 2023. https://www.techrepublic.com/article/top-programming-languages-employers-want/

            \bibitem{TopLangs3} “11 of the Most In-Demand Coding Languages,” Indeed Career Guide. https://www.indeed.com/career-advice/career-development/most-in-demand-coding-languages

            \bibitem{PyLibs} “Search results,” PyPI. https://pypi.org/search/

            \bibitem{NPM} npm, “npm | build amazing things,” Npmjs.com, 2019. https://www.npmjs.com/

            \bibitem{Yarn} Yarnpkg.com, 2019. https://yarnpkg.com/

            \bibitem{EasyLangs1} “10 Hardest and Easiest Programming Languages in 2024 - GUVI Blogs,” Mar. 01, 2023. https://www.guvi.in/blog/easiest-programming-languages-to-hardest-ranked/

            \bibitem{EasyLangs2} “The 10 Most Popular Coding Languages to Learn in 2023 | BestColleges,” www.bestcolleges.com. https://www.bestcolleges.com/bootcamps/guides/most-important-coding-languages/

            \bibitem{FP1} J. M. Fernandes, “Functional Programming: With Examples and Lots of Cats,” Medium, Sep. 29, 2021. https://medium.com/arctouch/thinking-functional-now-with-example-and-cats-8b9c2478b9af            

            \bibitem{FP2} GeeksForGeeks, “Functional Programming Paradigm - GeeksforGeeks,” GeeksforGeeks, Jan. 02, 2019. https://www.geeksforgeeks.org/functional-programming-paradigm/

            \bibitem{RefTransp1} “Referential transparency,” Wikipedia, Jan. 05, 2021. https://en.wikipedia.org/wiki/Referential\_transparency

            \bibitem{RefTransp2} Claudiu, “What is referential transparency?,” Stack Overflow, Oct. 17, 2008. https://stackoverflow.com/a/9859966

            \bibitem{ElixirList} beam-community, “elixir-companies/priv/companies at main · beam-community/elixir-companies,” GitHub, 2015. https://github.com/beam-community/elixir-companies/tree/main/priv/companies

            \bibitem{HaskellListReadme} erkmos, “GitHub - erkmos/haskell-companies: A gently curated list of companies using Haskell in industry,” GitHub, 2017. https://github.com/erkmos/haskell-companies?tab=readme-ov-file

            \bibitem{ElixirListReadme} beam-community, “GitHub - beam-community/elixir-companies: A list of companies currently using Elixir in production.,” GitHub, 2015. https://github.com/beam-community/elixir-companies

            \bibitem{HaskellList} “Haskell in industry - HaskellWiki,” Haskell.org, 2018. https://wiki.haskell.org/Haskell\_in\_industry

            \bibitem{OCamlCoq} coq, “GitHub - coq/coq: Coq is a formal proof management system. It provides a formal language to write mathematical definitions, executable algorithms and theorems together with an environment for semi-interactive development of machine-checked proofs.,” GitHub, Sep. 04, 2024. https://github.com/coq/coq?tab=readme-ov-file

            \bibitem{ErlangUses} F. Cesarini, “Companies Who Use Erlang,” Erlang Solutions, Sep. 11, 2019. https://www.erlang-solutions.com/blog/which-companies-are-using-erlang-and-why-mytopdogstatus/

        \end{thebibliography}

    \end{singlespace}

    \appendix
    \addcontentsline{toc}{chapter}{Appendix}
        
    \chapter{Proof Assistants}\label{app:ProofAssist}

Initially, my goal was to learn how to use proof assistants to assert properties about Proteus.
I began to learn how to write basic and simple proofs using Coq.
Coq was powerful enough to show properties of systems that were publishable.
Whilst learning how to use Coq, I was learning to use TLA+ which proved equally as powerful, but not suited for my use case.
TLA+ is used for simultaneous computing as well as state machines which seemed very appealing upon initial inspection, since Proteus uses state machines as well.
However, this didn't extend well into demonstrating parts of Proteus that were not related to state machines.
I then considered using a different proof assistant like Dafny, Lean, or Twelf to see which would be better suited for proving something about Proteus.
I continued learning how to use the proof assistants and aimed to apply them to whatever proof I would eventually create about Proteus.

The big idea hit me one day; To show that Proteus was Turing Complete.
It was one of the biggest things to show in theory and was able to utilize all my knowledge and experience from this discipline.
There was only 1 problem, I couldn't use any proof assistant.
This is because fundamentally, it is impossible to show Turing Completeness using a proof assistant.

Recall Turing Completeness means that the shown system is equivalently as powerful as a TM, see section \ref{sec:TC}.
Also, recall the Church-Turing Thesis, Theorem \ref{thm:CTT}, which means that the definition of a TM is to be able calculate any calculable function.
Additionally, recall the proof of showing a system is TC in section \ref{subsubsec:NotableEgsFormalLang}.
The purpose of a proof assistant is to assert that the given proof is valid, step by step until the goal is reached.
Therefore for all inputs for any proof assistant, they are decidable (assuming they are well-written).
Thus, they are not capable of satisfying the Church-Turing Thesis, Theorem \ref{thm:CTT}, and also cannot be as powerful as a TM.
As such, it is impossible to use a proof assistant to show Turing Completeness.
This is when I abandoned using a proof assistant and moved forward in favor of proving Proteus TC.
    
    \chapter{Code Segments}\label{app:CodeSegments}

In this appendix section, I hope to put some of the larger code segments so that there is less interruption when reading the text.

\end{document}